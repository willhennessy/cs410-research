\documentclass[../summaries.tex]{subfiles}

\begin{document}

\subsection{Citation}
Xing, Bo, and Zhangxi Lin. "The impact of search engine optimization on online advertising market." Proceedings of the 8th international conference on Electronic commerce: The new e-commerce: innovations for conquering current barriers, obstacles and limitations to conducting successful business on the internet. ACM, 2006.

\subsection{Summary}
In the past decade, search engine advertising has grown into a multibillion dollar industry. This money stems from two distinct forms of advertisement:  paid placement and Search Engine Optimization (SEO). Paid placement is implemented by the search engines themselves (i.e. Google sponsored searches) and are discussed at length in other papers. This paper examines the industry of Search Engine Optimization. A multitude of SEO firms have cropped up around the search industry to harness its economic prosperity. Businesses around the world contract these SEO firms to optimize their business website and boost its ranking in search engine results.

This is considered an alternate form of advertising because businesses spend money to increase their brand visibility to consumers. While paid placement has its own advantages, the unique advantage of SEO is that it attracts the user by listing the website as an organic search result. Multiple studies have reported that users significantly prefer clicking on an organic link rather than an advertisement.

The overall impact of SEO firms on the consumer and businesses is examined in this paper. Overall, SEO introduces additional noise in the ranking algorithm. SEO's artificial inflation of web page ranking sometimes promotes genuinely useful sites, but other times it spams search results with undesirable web pages that have an inflated ranking. The authors define "algorithm robustness" as a search engine's ability to exclude this noise. A highly robust search engine will improve customer satisfaction with search results, but decrease the effectiveness of SEO advertising efforts.

The authors combine algorithm robustness with algorithm effectiveness (approximated as user satisfaction) to create a model that estimates the impact of SEOs on the advertising market. The study makes a few interesting observations. First, SEO firms depend on advertisers' willingness to pay for online advertisements. If global businesses shifted to a new form of advertising (i.e. augmented reality posters on the streets of New York) then SEOs would be left destitute. Second, search engines can and do combat the manipulation of SEO firms by investing in algorithm robustness. This increases search engine profits and diminishes the effect of SEO, but SEO firms still provide an impactful advertising strategy for business owners. These findings correspond with those in other papers; business owners should pay money to earn a high ranking in search results because it dramatically increases click-throughs and conversions on their business website.

\end{document}
