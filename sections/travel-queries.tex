\documentclass[../summaries.tex]{subfiles}

\begin{document}

\subsection{Citation}
Xiang, Zheng, and Bing Pan. "Travel queries on cities in the United States: Implications for search engine marketing for tourist destinations." Tourism Management 32.1 (2011): 88-97.

\subsection{Summary}
One very common use for search engines is travel planning. Studies have shown that search engines are the number one information source for American families who are planning a vacation. This paper analyzes search data from all around the United States to provide some insights on the tourism industry. Here are the conclusions they reached:

A couple of tourism-related metrics can be derived through search data. First, the "touristic" level of a city can be evaluated by observing the ratio of tourism-related queries to total queries performed in a city. Second, the overall popularity of a tourist destination can be determined simply through measuring the pure volume of tourism-related queries about the destination. This provides a strong link between the search economy online, and the tourism industry. 

Search sessions containing travel queries are relatively short, usually consisting of one to three queries in total. This is important to marketers, as it means that there are not too many opportunities to make an impression on a user. These travel queries usually make reference to both "core" keywords, most of which relate to transportation and accommodation, and context-dependent "specialty" keywords. In this case, the context is contingent on both the size of the destination, and the "touristic" level of the city. For large metropolitan cities, searches for maps, parks, and attractions dominated the specialty keywords. For middle to small sized destinations, the specialty keywords were found to focus significantly more on specific tourist attractions. This means that search engine marketing strategies will vary significantly depending on the destination.

The study shows that destination marketing organizations (DMOs) can benefit significantly from monitoring search data. Despite the fact that accommodation related keywords are the most commonly searched online by potential tourists, promoting accommodation is not of the highest priority to DMOs. Along with this, DMOs can use search data to look for trends that vary based off of specific variables, for example, weather or season.

\end{document}
