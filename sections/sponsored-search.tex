\documentclass[../summaries.tex]{subfiles}

\begin{document}

\subsection{Citation}
Ghose, Anindya, and Sha Yang. "An empirical analysis of sponsored search performance in search engine advertising." Proceedings of the 2008 International Conference on Web Search and Data Mining. ACM, 2008.

\subsection{Summary}
Businesses have many choices for online advertising. In search engines, the predominant ad form is "sponsored searches" in which the advertiser pays a fee to appear next to organic search results. For example, a consumer who searches "digital camera" is likely to see advertisements from Kodak in the search results because the company paid for ad space next to the "digital camera" keyword. A company can purchase any number of different keywords, so choosing which keywords to purchase and how much to pay for them will dramatically impact the success of a marketing campaign.

This paper empirically models the relationship between keywords, click-through rates, and conversion rates by analyzing a real world dataset from a large retail company. The data is fit into a hierarchical Bayesian framework and then run through Markov Chain Monte Carlo methods to estimate the results. The goal is to examine how click-through rates and conversion rates are affected by the following three characteristics of a keyword:

\begin{enumerate}
\item \textbf{Brand:}  does the keyword contain a brand name? (i.e. Kodak, Nikon)
\item \textbf{Retailer:}  does the keyword contain a retailer name? (i.e. Best Buy, Amazon)
\item \textbf{Length:}  how many words are in this keyword? (i.e. "camera" vs. "black dslr HD camera")
\end{enumerate}

\textbf{Impact on Click-Through Rates}

The first interesting result is that keyword advertisements with retailer-specific terms cause a 28.31\% increase in click-through rates. This confirms the belief that users actually enjoy advertisements when they search for a specific retailer. In a surprising contrast, keywords containing a specific brand have no statistically significant effect on click-through rates. Finally, the length of the keyword is inversely related to click-through rates; an increase in length by one word decreases the click-through rates by 6.6\%.

\textbf{Impact on Conversion Rates}

Keywords containing a brand-specific term experience an increase in conversion rates by 21.35\%. This finding is important to businesses because it confirms that branded keywords are extremely valuable to the advertiser. Conversion rates are also impact by click-through rates. If the click-through rate is increased from the minimum (0.0) to the maximum (1.0), the conversion rate increases by 63.31\%. However, this increase is dwarfed by the impact of rank:  an ad that moves from the worst rank (bottom of page) to the best rank (top of page) yields an increased conversion rate of 99.97\%. Finally, the quality of a business's landing page (as scored by Google on a 1-10 point scale) corresponds with the conversion rate. Using this metric, the authors found that a one point increase in landing page quality corresponded to a 22.5\% increase in conversion rate.

Businesses can use these statistics to choose the optimal budget for keywords based on brand, retailer, and length.

\end{document}
