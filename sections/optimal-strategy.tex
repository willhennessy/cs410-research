\documentclass[../summaries.tex]{subfiles}

\begin{document}

\subsection{Citation}
Sen, Ravi. "Optimal search engine marketing strategy." International Journal of Electronic Commerce 10.1 (2005): 9-25.

\subsection{Summary}
This study was written in 2005--a time when businesses were still trying to figure out how to leverage search engine technology as a marketing tool. The paper starts by listing several different kinds of search engine marketing (SEM) strategies:

\begin{enumerate}
\item Keyword-related banner advertisements
\item Paid submission for regular updates: Businesses can pay a one-time fee to a search engine provider to guarantee that their site is reviewed within a short time frame, so it can show up in search results as soon as possible.
\item Search engine optimization (SEO): Businesses can influence their site's visibility in a search engine's natural results by modifying their site code.
\item Paid placements: Businesses can pay a search engine provider in order to be listed in their "sponsored" section in search result pages.
\end{enumerate}

At the time of publishing, the majority of costs associated with SEM (around 82 percent) went towards paid placement campaigns. Despite this, buyers using search engines to find information generally only follow links that are found in the editorial section of the results page (where the natural results are found). This implies a couple of things. First, that SEO is probably the most effective marketing strategy. If a business can modify its site such that it naturally appears at the top of results, the conversion rate would be significantly higher than what could be achieved from any of the other SEM strategies. Second, it implies that the quality of SEO solutions in 2005 was insufficient for producing quality results.

The study ran an analysis on various SEM strategies, and came to a few conclusions:

Even though SEM was a high-growth part of online marketing at the time, it was not the dominant form. This was due to advertisers having doubts about its effectiveness. The analysis discovered that SEM was worth investing in if a business's market is characterized by: buyers who have low search intensity (e.g., buyers with opportunity cost of time), a product sold by many other providers (e.g., computers), or if the product is of low value (e.g., books). For businesses involved in niche markets (e.g. vintage cars), SEM was not worth investing in. This is because the number of competing businesses online was never high enough for SEM to be worth the cost.

Economically speaking, the analysis found that SEO was never a part of a business's equilibrium strategy. When there were a lot of competing companies, the probability of appearing in top results of a search engine was so low, that SEO couldn't do much to improve the situation. In this case it made more sense to pay for sponsored links, which were guaranteed to appear to users. When there were not many competing companies, investment in SEO was redundant, since the probability of appearing in top results was high enough to begin with.

Concerns were raised about paid placement strategies. Their effectiveness is determined by click-through rates, which can be artificially boosted by either people or software.
	
\end{document}
